\documentclass[final]{beamer}
\usepackage[orientation=portrait,size=a4,scale=1.0]{beamerposter}
\usetheme{gemini}
\usepackage{lipsum}
\usecolortheme{nott}
\usepackage{graphicx}
\usepackage[spanish]{babel}
\usepackage{newtxtext,fourier} 
\usepackage{tikz}
\usepackage{pgfplots}
\pgfplotsset{compat=1.14}
\usepackage{amsmath, amsfonts, amsthm, amssymb}
\newcommand\T{\ensuremath{\mathbb{T}}}
\newcommand\N{\ensuremath{\mathbb{N}}}
\newcommand\R{\ensuremath{\mathbb{R}}}
\newcommand\Z{\ensuremath{\mathbb{Z}}}
\newcommand\Q{\ensuremath{\mathbb{Q}}}
\newcommand\C{\ensuremath{\mathbb{C}}}
\newlength{\sepwidth}
\newlength{\colwidth}
\setlength{\sepwidth}{0.025\paperwidth}
\setlength{\colwidth}{0.45\paperwidth}
\newcommand{\separatorcolumn}{\begin{column}{\sepwidth}\end{column}}
\newtheorem{teor}{Teorema}
\newcommand{\defi}[1]{\textbf{\emph{#1}}} 
\title{\huge Sobre el teorema de los números primos en progresiones aritmética}

\author{\Large Mateo Andrés Manosalva Amaris\\Trabajo dirigido por John Jaime Rodriguez\\Departamento de Matemáticas, Universidad Nacional de Colombia}
\logoleft{
  \begin{tabular}{@{}c@{}}
    \hspace{1.2cm}\includegraphics[scale=0.05]{logo.jpg} \\
  \end{tabular}
}
\begin{document}
\Large
\begin{frame}[t,fragile]
\begin{columns}[t]
\separatorcolumn

\begin{column}{\colwidth}

\begin{block}{Resumen}
El teorema de los números primos nos dice que en el límite, el cociente $\dfrac{\pi(x)\log x}{x}$ tiende a 1, es decir, que $\pi(x)\thicksim \dfrac{x}{\log x}$ donde $\pi(x)$ es la función contadora de primos. En progresiones aritméticas $a+kq$ con $(a,q)=1$, tenemos que $\pi(a,q,x)$; la función contadora restringida a la progresión, tiene el comportamiento asintótico $\pi(a,q,x)\thicksim \dfrac{x}{\phi(q)\log x}$, es decir, los primos se distribuyen uniformemente en las clases de residuos módulo $q$. En este trabajo se presentará la prueba de este resultado y las ideas subyacentes. Para esto, haremos uso de la teoría Tauberiana, lo que nos permitirá presentar una prueba detallada y corta, que se seguirá estudiando la no nulidad de $L(\chi,s)$ y algunas propiedades de los caracteres y series de Dirichlet.
\end{block}

\begin{block}{El Teorema de Dirichlet}

Desde los tiempos de Euclides se sabe que existen infinitos números primos, sin embargo no se conocía mucho mucho sobre su distribución,
\begin{itemize}
    \item  ¿Existen infinitos números primos de la forma $a+kn$?
    \item ¿Cómo se distribuyen los números primos en cada  una de las clases de equivalencia módulo $n$?.
\end{itemize}
El argumento de Euclides provenía de ver que si existen finitos números primos, digamos $p_1,\ldots,p_n$, entonces $p_1p_2\cdots p_n+1$ no es divisible por ningún primo $p_i$.\\
\vspace*{0.3cm}
\defi{Teorema (Dirichlet)}
Dados $a$ y $d$ primos relativos, existen infinitos primos de la forma
    $$a, a+d,a+2d,a+3d,...$$
\defi{Teorema (Euler)} La suma $\displaystyle  \sum_{p}\dfrac{1}{p}$ es divergente.\\

La idea de Euler consiste en explotar la identidad 
$$\prod_p \left(1-\dfrac{1}{p^s}\right)^{-1}=\sum_{n=1}^{\infty}\dfrac{1}{n^s},\quad  \Re(s)>1$$
de donde obtiene que 
    $$
\log (\zeta(s))=\sum_p^{\infty}\left(\displaystyle\sum_{k=1}^{\infty} \dfrac{1}{k(p)^{k s}}\right)=\sum_p
\dfrac{1}{p^s}+\sum_{p}\left(\sum_{k=2}^{\infty}\dfrac{1}{kp^{ks}}\right), \quad \Re(s)>1$$
\end{block}
    
\begin{alertblock}{La idea de Dirichlet}

Sea $f(n)$ la función característica de la progresión aritmética, es decir
$$
f(n)=\left\{\begin{array}{lll}
1, & n \equiv a & \pmod{m} \\
0, & n \not \equiv a & \pmod{m}
\end{array}\right.
$$
en el caso de que $f(n)$ sea completamente multiplicativa tendríamos un producto de Euler
$$
\sum_{n=1}^{\infty} \frac{f(n)}{n^s}=\prod_p\left(1-\frac{f(p)}{p^s}\right)^{-1}, \quad \Re(s)>1
$$
y así por argumentos análogos a los de Euler se tendría que
$$\log \left(\sum_{n=1}^{\infty} \frac{f(n)}{n^s}\right)=\sum_{p \equiv a\bmod{m}} \frac{1}{p^s}+O(1)$$
Lamentablemente, $f(n)$ generalmente no es multiplicativa.\\
  \end{alertblock}

\end{column}

\separatorcolumn

\begin{column}{\colwidth}
\phantom{xd}\\\vspace*{0.6cm}
El camino para probar este teorema es ver que $\displaystyle\psi(x)=\sum_{n\leq x} \Lambda(n)\thicksim x,$ donde $\Lambda(n)$ es la función de Von Mangolth, $\displaystyle\Lambda(n)=\begin{cases}
    \log(p) &\text{si } n=p^k, k\geq 1\\
    0 &\text{e.o.c}
\end{cases}$
\begin{block}{Teoremas tauberianos}
\defi{Proposición: }Sea $\displaystyle \sum_{n=0}^{\infty} a_n x^n$, $x\in \R$ una serie de potencias centrada en $0$ y con radio  de convergencia $1$, si
$$\sum_{n=0}^{\infty} a_n=A,\text{ entonces } \lim_{x \to 1^-}\sum_{n=0}^{\infty} a_n x^n=A.$$
Los teoremas tauberianos son recíprocos condicionales del teorema de Abel.\\
\defi{Proposición (Tauber, 1897)} Sea $f(x)=\displaystyle\sum_{n=0}^{\infty} a_n x^n$ una serie de potencias que converge absolutamente para $|x|<1$.
Si $\lim_{x \rightarrow 1^{-}} f(x)=A$ y se cumple la condición $a_n=o\left(\dfrac{1}{n}\right)$, entonces $\displaystyle\sum_{n=0}^{\infty}a_n=A$.
\end{block}
\begin{exampleblock}{Teorema de Wiener-Ikehara}
Sean $a_n \geq 0$ y $F(s)=\displaystyle\sum_{n=1}^{\infty} \frac{a_n}{n^s}$ una serie absolutamente convergente. Supongamos que se cumplen las siguientes condiciones:
\begin{itemize}
\item La función $F(s)$ se extiende a una función analítica en la región $\Re(s) \geq 1$ con un único polo simple en $s=1$, cuyo residuo es $1$.\\
\item $A(x)=\displaystyle \sum_{n \leq x} a_n=O(x)$.
\end{itemize}
Entonces, se tiene que $A(x)=x+o(x) \text { cuando } x \rightarrow \infty \text {. }
$
\end{exampleblock}
Aplicando el teorema anterior a la serie de Dirichlet $\displaystyle-\dfrac{\zeta^{\prime}(s)}{\zeta(s)}=\sum_{n=1}^{\infty}\dfrac{\Lambda(n)}{n^s}$, se obtiene el TNP, como corolario de
$$\zeta(1+it)\neq 0, \text{ para todo } t\neq 0.$$
\defi{Teorema (Korevaar y Zagier)} Para $t \geq 0$, sea $f(t)$ una función acotada y localmente integrable y sea $$g(s):=\displaystyle\int_0^{\infty} f(t) e^{-s t} d t,$$
para $\Re(s)>0$. Si $g(s)$ tiene continuación analítica a $\Re(s) \geq 0$, entonces $\displaystyle\int_0^{\infty} f(t) d t$ existe y es igual a $g(0)$.
La prueba consiste en estimar la integral
$$I_{C}=\frac{1}{2\pi i}\int_{C}\left(g(s)-g_T(s)\right)e^{sT}\left(1+\frac{s^2}{R^2}\right)\frac{1}{s}ds=g(0)-g_T(0),$$
donde $C$ es el siguiente contorno

Obtenemos \[
g(0) = \lim _{T \rightarrow \infty} g_T(0).
\]
\defi{Teorema (Korevaar y Zagier) }Sean $a_n\geq 0$ y $A(x)=\displaystyle\sum_{n\leq x} a_n$, si  la integral $\displaystyle\int_1^{\infty}\frac{A(x)-x}{x^2}dx$
converge, entonces $A(x)\thicksim x$\\
\vspace*{0.2cm}
Aplicando lo anterior y sabiendo que 
\[
\sum_{n=1}^{\infty}a_nn^{-s} - \frac{s}{s-1} = s \int_1^{\infty} \frac{A(t)-t}{t^{s+1}} d t,
\]
se obtiene una prueba del teorema tauberiano.
\end{column}

\separatorcolumn

\end{columns}
\end{frame}

\newpage
\thispagestyle{empty}
\vspace*{-1.5cm}

\begin{frame}[t,fragile]
\begin{columns}[t]
\separatorcolumn
\begin{column}{\colwidth}

\begin{block}{La familia pepa}
\lipsum[1]
\end{block}
\lipsum[2-6]

\end{column}

\separatorcolumn

\begin{column}{\colwidth}
\lipsum[1-7]
\end{column}

\separatorcolumn

\end{columns}
\end{frame}


\end{document}
