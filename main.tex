\documentclass[final]{beamer}
\usepackage[T1]{fontenc}
\usepackage{lmodern}
\usepackage[orientation=portrait,size=a0,scale=1.0]{beamerposter}
\usetheme{gemini}
\usecolortheme{nott}
\usepackage{graphicx}
\usepackage[spanish]{babel}
\usepackage{booktabs}
\usepackage{tikz}
\usepackage{pgfplots}
\pgfplotsset{compat=1.14}
\usepackage{anyfontsize}
\usepackage{amsmath, amsfonts, amsthm, amssymb}
\newlength{\sepwidth}
\newlength{\colwidth}
\setlength{\sepwidth}{0.025\paperwidth}
\setlength{\colwidth}{0.45\paperwidth}
\newcommand{\separatorcolumn}{\begin{column}{\sepwidth}\end{column}}
\newtheorem{teor}{Teorema}
\usepackage{lipsum}

\title{Sobre el teorema de los números primos en progresiones aritmética}

\author{Mateo Andrés Manosalva Amaris\\Trabajo dirigido por John Jaime Rodriguez\\Departamento de Matemáticas, Universidad Nacional de Colombia}
\logoleft{\includegraphics[scale=0.22]{logo.jpg}}

\begin{document}

\begin{frame}[t,fragile]
\begin{columns}[t]
\separatorcolumn

\begin{column}{\colwidth}

  \begin{block}{Resumen}

\lipsum[1]

  \end{block}

\begin{block}{El teorema de Dirichlet}
    \lipsum[2]
\end{block}
    
  \begin{alertblock}{Bloque de color}

    \lipsum[1-5] 
  \end{alertblock}

 \begin{block}{The Goldie Condition}
    \lipsum[1-4]
  \end{block}

\end{column}

\separatorcolumn

\begin{column}{\colwidth}
  \begin{block}{Aquí comienza la otra columna}
    \lipsum[1-4]
  \end{block}
  
  

 

  \begin{block}{References}

    \nocite{*}
    \footnotesize{\bibliographystyle{plain}\bibliography{poster}}

  \end{block}

\end{column}
\separatorcolumn



\end{columns}
\end{frame}

\end{document}
